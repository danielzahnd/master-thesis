\begin{abstract}
\normalsize
As a source of not only all energy needed on earth to enable plants to grow or weather phenomena to form, but also of inspiration and fascination, the sun and its underlying physics has ever been of utmost interest. There are certain solar phenomena such as solar flares or coronal mass ejections, which pose potential threats to technical endeavors on earth and also in space, such as power grids, spacecraft and ultimately humans serving as astronauts in space. Therefore, in analogy to meteorological forecasts on earth, there is a desire to be able to forecast solar flares and coronal mass ejections; this field of research is commonly called space weather. In order to make predictions of solar flares or coronal mass ejections, certain precursors in the solar atmosphere causally related or correlated to these phenomena have to be identified and studied. Aiming for this goal, inversions of so-called atmospheric models of stars are needed to obtain and investigate those physical parameters and to assess, whether some could be identified as precursors or not. Many stellar atmospheres are not invertible in an analytical way, which is to say that inversions of more complex stellar models are only possible by lots of computational power. Therefore, an alternative, in terms of computational power less demanding way of inverting stellar atmospheres is explored in this thesis. In the thesis at hand, normalizing flows are applied to observations of the Swedish Solar Telescope (SST) in combination with the Milne-Eddington model as the sun's atmospheric model.
\end{abstract}