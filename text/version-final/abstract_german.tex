\selectlanguage{german}
\begin{abstract}
\normalsize
Die Sonne ist nicht nur eine Quelle für die Energie, die auf der Erde benötigt wird, um Pflanzen wachsen zu lassen und damit sich Wetterphänomene bilden können, sondern auch eine inspirierende und faszinierende Erscheinung. Bestimmte solare Phänomene wie Sonneneruptionen oder koronale Massenauswürfe können potenzielle Gefahren für technische Objekte auf der Erde und im Weltraum darstellen, wie beispielsweise für Stromnetze und Raumfahrzeuge. Auch Menschen im Weltraum müssen vor schädlicher ionisierender Strahlung bestmöglich geschützt werden können. Ähnlich wie bei Wettervorhersagen auf der Erde besteht daher das Bedürfnis, dasselbe auch hinsichtlich der Sonne zu tun; also für Mensch und Technik potentiell gefähliche Sonnenphänomene anhand physikalischer Merkmale der Sonne vorherzusagen. Dieser Bereich der Forschung ist als ``space weather'' bekannt. Um solche Vorhersagen treffen zu können, müssen die mit Sonneneruptionen und koronalen Massenauswürfen assoziierten Merkmale in der Sonnenatmosphäre identifiziert und studiert werden. Hierfür werden sogenannte Inversionen atmosphärischer Modelle von Sternen benötigt, um von Beobachtungen der Sonnenatmosphäre auf die in ihr herrschenden physikalischen Bedingungen rückschliessen zu können. Manch ein Modell einer Sternatmosphäre ist jedoch nicht analytisch invertierbar (umkehrbar), sodass \lk{numerische} Inversionen komplexer Modelle viel Rechenleistung erfordern. Daher wird in dieser Arbeit eine weniger rechenintensive Alternative zur Inversion stellarer Atmosphären erforscht: Unter der Annahme des Milne-Eddington Modells für die solare Atmosphäre werden sogenannte ``normalizing flows'' auf Beobachtungen des Swedish Solar Telescope (SST) angewandt.
\end{abstract}
\selectlanguage{english}